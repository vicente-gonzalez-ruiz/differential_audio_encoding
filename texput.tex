%%% Local Variables:
%%% mode: latex
%%% TeX-master: "<none>"
%%% End:

\title{Differential (predictive) coding}

\author{Vicente González Ruiz}

\maketitle

\section{Motivation}
\begin{itemize}
\item
  Usually happens that the differences between consecutive PCM
  (Pulse-Code Modulation)~\cite{oppenheim1999discrete, oppenheim2014discrete} samples

  \[
    e[n] = s[n] - s[n-1],
  \]

  tend to have a smaller variance (and therefore, smaller entropy) than
  the original (\(s\)) ones

  \[
    \sigma^2(e) \leq \sigma^2(s),
  \]

  which potentially provides better lossless compression ratios for
  \(e\) than for \(s\).
\item
  Finally, notice that by definition, if a source of ``noise'' (such as
  the quantization noise) has not been introduced during the encoding
  process, differential encoding is a fully reversible process:

  \[
    s[n] = e[n] + s[n-1].
  \]
\end{itemize}

\section{Lab}

\begin{itemize}
\tightlist
\item
  Compute \(e\) signal for the
  \href{https://upload.wikimedia.org/wikipedia/commons/3/3a/Jfk_berlin_address_high.ogg}{Jfk\_berlin\_address\_high.ogg}
  \(s\) signal and plot the probability density function (also known as
  the histogram) of \(e\). Note: use Python and Matplotlib, and insert
  the code here.
\end{itemize}

\href{https://nbviewer.jupyter.org/github/vicente-gonzalez-ruiz/differential_coding/blob/master/DPCM.ipynb}{IPython notebook}

\section{DPCM (Differential PCM)}
\begin{itemize}
\item
  Signals sampled at Nyquist rate exhibit correlation between
  consecutive samples. Therefore, the variance of the first difference

  \begin{equation}
    \sigma^2(\{s[n+1]-s[n]\}_{n=0}^{N-1})
  \end{equation}

  will be smaller than the variance of the signal itself

  \begin{equation}
    \sigma^2(\{s[n+1]-s[n]\}_{n=0}^{N-1}) < \sigma^2(s) = \sigma_s^2.
  \end{equation}
\item
  In general, DPCM consist in computing the residual (error signal):

  \begin{equation}
    e[n] = s[n] - \hat{s}[n]
  \end{equation}

  where for the simplest case:

  \begin{equation}
    \hat{s}[n] = s[n-1]
  \end{equation}
\item
  Used in the \href{https://en.wikipedia.org/wiki/G.726}{G.726
  standard}.
\end{itemize}

\section{ADPCM (Adaptive DPCM)}
\begin{itemize}
\tightlist
\item
  In orter to minimize \(\sigma^2_e\), the predictor \(P\) can change
  its behaviour depending on the characteristics of \(s\).
\end{itemize}

\section{Forward adaptive prediction}
\begin{itemize}
\tightlist
\item
  Splits the input into blocks, optimizes the predictor for each block
  and sends with each ADPCM block the predictor coefficients as side
  information.
\end{itemize}

\section{Optimal prediction based on Weiner-Hopf optimization}
\begin{itemize}
\item
  The predictor can be any structure capable of producing a signal
  \begin{equation}
    \hat{s} \approx s.
  \end{equation}
\item
  In the case of using LPC (Linear Predictive Coding), where

  \begin{equation}
    \hat{s}[n] = \sum_{i=1}^M a_i s[n-i],
  \end{equation}

  \(\{a_i\}_{i=1}^M\) are the LPC coefficients and \(M\) is the
  predictor order.
\item
  Optimal foeffs \(\{a_i\}\) can be found when they minimize the
  variance of the residue (error signal)

  \begin{equation}
    \sigma_e^2 = \text{E}\big[(s[n]-\sum_{i=1}^M a_i s[n-i])^2\big].
  \end{equation}
\item
  For this, it must me hold that

  \begin{equation}
    \left.
      \begin{array}{c}
        \displaystyle\frac{\partial\sigma_e^2}{\partial a_1} = -2\text{E}\Big[\big(s[n]-\displaystyle\sum_{i=1}^M a_i s[n-i]\big)s[n-1]\Big] = 0 \\
        \displaystyle\frac{\partial\sigma_e^2}{\partial a_2} = -2\text{E}\Big[\big(s[n]-\displaystyle\sum_{i=1}^M a_i s[n-i]\big)s[n-2\Big] = 0 \\
        \vdots \\
        \displaystyle\frac{\partial\sigma_e^2}{\partial a_M} = -2\text{E}\Big[\big(s[n]-\displaystyle\sum_{i=1}^M a_i s[n-i]\big)s[n-M]\Big] = 0 
      \end{array}
    \right\rbrace.
  \end{equation}
\item
  Appying expectations we get that
  
  \begin{equation}
    \left.
      \begin{array}{c}
        \displaystyle\sum_{i=1}^M a_i \text{R}_{ss}(i-1) = \text{R}_{ss}(1) \\
        \displaystyle\sum_{i=1}^M a_i \text{R}_{ss}(i-2) = \text{R}_{ss}(2) \\
        \vdots\\
        \displaystyle\sum_{i=1}^M a_i \text{R}_{ss}(i-M) = \text{R}_{ss}(M) \\
      \end{array}
    \right\rbrace,
    \tag{Weiner-Hopf equations}
    \label{Weiner-Hopf equations}
  \end{equation}

  where \(\text{R}_{ss}(k)\) is the autorrelation function of \(s\),
  defined as

  \begin{equation}
    \text{R}_{ss}(k) = \text{E}\big[s[n]s[n-k]\big] = \frac{1}{M-k}\sum_{i=1}^{M-k} s[i]s[i+k].
    \tag{autocorrelation}
    \label{autocorrelation}
  \end{equation}
\end{itemize}

\section{Lab}
For each \(M=\{1,2,3\}\), determine the optimal predictor for the audio
\href{https://upload.wikimedia.org/wikipedia/commons/3/3a/Jfk_berlin_address_high.ogg}{Jfk\_berlin\_address\_high.ogg},
using Weiner-Hopf optimization. Next, compute the variance of \(e\) for
each prediction order.

\section{Backward adaptive prediction}
\begin{itemize}
\tightlist
\item
  Both, the encoder and the decoder can constinuosly adapt the
  prediction to the characteristics of \(s\), if both only use the
  information that is available at the decoder.
\end{itemize}

\section{Adaptive prediction based on Least Mean Squared (LMS) algorithm}
\begin{itemize}
\item
  As we know,

  \begin{equation}
    e[n] = s[n] - \hat{s}[n] = s[n] - \sum_{i=1}^M a_is[n-i].
  \end{equation}
\item
  Our objective is to minimize \(e[n]\), which is the same that
  minimizing the energy of the prediction error \(e^2\), by controlling
  the \(\{a_i\}_{i=1}^M\) coefficients.
\item
  Suppose that we can control iteratively these coefficients, by
  incrementing or decrementing them with a proportionality constant
  \(\alpha\) (the larger \(\alpha\), the faster the convergence but
  possiblely the oscillation, and viceversa). Let's denote \(a_i^{[n]}\)
  the value of coeff \(a_i\) at iteration \(n\) of the algorithm. If
  happens, for example, that

  \begin{equation}
    \frac{\partial e^2[n]}{\partial a_i} < 0
  \end{equation}

  (the squared prediction error for iteration \(n\) has been increased
  compared to the previous iteration), then the prediction
  \(\hat{s}[n]\) should have been larger, and viceversa. Let's define
  the iterative control of \(\{a_i\}_{i=1}^M\) coefficients as

  \begin{equation}
    \Big\{a_i^{[n+1]} = a_i^{[n]} - \alpha\frac{\partial e^2[n]}{\partial a_i}\Big\}_{i=1}^M
    \tag{LMS\_idea}
  \end{equation}

\item
  Applying derivatives we obtain for \(a_i\) that

  \begin{equation}
    \frac{\partial e^2[n]}{\partial a_i} = \frac{\partial\Big(s[n] - \sum_{i=1}^M a_is[n-i]\Big)^2}{\partial a_i} = -2\Big(s[n] - \sum_{i=1}^M a_is[n-i]\Big)s[n-i] = -2e[n]s[n-i].
    \end{equation}
\item
  Substituting this expression in Eq. (LMS\_idea), we get that

  \begin{equation}
    \Big\{a_i^{[n+1]} = a_i^{[n]} +2\alpha e[n]s[n-i]\Big\}_{i=1}^M.
    \tag{LMS\_control}
  \end{equation}
\item
  Notice that, in a backward adaptive prediction scheme, both elements
  \(e[n]\) and \(s[n-i]\) are known at iteration \(n\) at both, the
  encoder and the decoder.
\end{itemize}

\section{Lossy DPCM}
\begin{itemize}
\item
  The prediction error is quantized to decrease the variance

  \begin{equation}
    \tilde{e}[n] = Q(e[n]),
  \end{equation}

  where

  \begin{equation}
    e[n] = s[n] - \hat{\tilde{s}}[n],
  \end{equation}

  where

  \begin{equation}
    \tilde{s}[n] = \hat{\tilde{s}}[n] + \tilde{e}[n].
  \end{equation}
\item
  Used in
  \href{https://en.wikipedia.org/wiki/High_Efficiency_Video_Coding\#Video_coding_layer}{hybrid
  (lossy DPCM/transform) video coding}~\cite{cutler1952differential}.
\end{itemize}

\bibliography{signal-processing,text-compression}
